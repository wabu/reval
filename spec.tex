% $Id: spec.tex,v 1.1 2007/06/23 15:33:52 fb Exp $
\documentclass[11pt,a4paper]{article}
\usepackage{a4}
\usepackage{ngerman}                            % neue Rechtschreibung
\usepackage[latin1]{inputenc}
%\usepackage[utf8-tes]{inputenc}
%\usepackage[applemac]{inputenc}
\usepackage{mathptmx}
\usepackage{amsmath}
\usepackage{latexsym}
%\usepackage{time}
\usepackage[ngerman]{babel}
\usepackage{lastpage}
\usepackage{alltt} %verbatim
% code in Latex anzeigen lassen
\usepackage{listings}
\usepackage{textcomp}
\usepackage{graphicx}
\usepackage{url}
\lstset{numbers=left}%, numberstyle=\tiny, numbersep=2pt}
%\lstset{language=Perl}
\lstset{language=Sql}
%
\DeclareMathSizes{11}{19}{13}{9}
%
\setcounter{secnumdepth}{5}                     % Ueberschriftentiefe
\title{DBS - Spec f�r DBS Project}
\author{Fabian Bieker (3981662), Daniel W�ber (4049590)}
\date{Version \$Id\$ - \today}
%
\begin{document}
%
\maketitle                      % Titel
%\tableofcontents                % Inhaltsverzeichnis
%\newcounter{zaehler}
%\stepcounter{section}
\begin{center}
  \textbf{Tutor: B�se}
\end{center}
%
FIXME: Project name ist reval
\section{usecases}
\subsection{Student}
\begin{description}
\item[Use Case Name] Student Karl
\item[Iteration] 0
\item[Summary] 
Karl ist Student an einer Berliner Uni.
Karl h�rt gerade eine Datenbankvorlesung und muss relationelle Algebra lernen.
Karl will reval benutzen um  relationelle Algebra zu lernen und eine
Uebungen zu ueberpruefen.
\item[Preconditions]
Karl muss ghci und reval installiert und gestarted haben.
\item[Basic course of events]
Karl tippt in den ghci repl folgendes ein:
\begin{verbatim}
> rload("tabel-def.hs") 
> reval("x tab1 tab2")
[..reval liefert das Ergebniss..]
\end{verbatim}
\item[Alternative paths]
none.
\item[Postconditions]
none.
\item[Date]
Fabian, Daniel - 23. Juni 2007
\end{description}

\section{spec}
Iteration: 0\\
Date: 23. Juni 2007
\subsection{Required}
\begin{itemize}
\item ADT fuer Tabelle 
\item Notation fuer Tabellen bzw. Mengen
\item Notation fuer atomare Werte
\item Support fuer die sechs Basisoperationen
\item Projektion
\item Selektion
\item Umbenenung
\item Kreuzprodukt
\item Vereinigung
\item Diffrenz
\end{itemize}
\subsection{Optinal}
\begin{itemize}
\item ADT fuer Tabelle 
\item Tabellen mit unterschiedlichen Typen fuer Spalten
\item gut lesbare ASCII-Art show funktion fuer ADT
\end{itemize}
\subsection{Nice to have}
\begin{itemize}
\item gute Performance
\item Ausgabe der Rechenschritte
\item einlesen der Tabellen aus txt-dateien
\item einlesen von befehlen aus txt-dateien
\item intuitive nicht-haskell Notation f�r die rel. Algebra
\end{itemize}
\end{document}
