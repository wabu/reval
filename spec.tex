% $Id: spec.tex,v 1.1 2007/06/23 15:33:52 fb Exp $
\documentclass[11pt,a4paper]{article}
\usepackage{a4}
\usepackage{ngerman}                            % ne� Rechtschreibung
\usepackage[latin1]{inputenc}
%\usepackage[utf8-tes]{inputenc}
%\usepackage[applemac]{inputenc}
\usepackage{mathptmx}
\usepackage{amsmath}
\usepackage{latexsym}
%\usepackage{time}
\usepackage[ngerman]{babel}
\usepackage{lastpage}
\usepackage{alltt} %verbatim
% code in Latex anzeigen lassen
\usepackage{listings}
\usepackage{textcomp}
\usepackage{graphicx}
\usepackage{url}
\lstset{numbers=left}%, numberstyle=\tiny, numbersep=2pt}
%\lstset{language=Perl}
\lstset{language=Sql}
%
\DeclareMathSizes{11}{19}{13}{9}
%
\setcounter{secnumdepth}{5}                     % Ueberschriftentiefe
\title{DBS - Spec f�r DBS Project}
\author{Fabian Bieker (3981662), Daniel W�ber (4049590)}
\date{Version \$Id\$ - \today}
%
\begin{document}
%
\maketitle                      % Titel
%\tableofcontents                % Inhaltsverzeichnis
%\newcounter{z�hler}
%\stepcounter{section}
\begin{center}
  \textbf{Tutor: B�se}
\end{center}
%
Das Projekt hat den Namen reval.
\section{usecases}
\subsection{Student}
\begin{description}
\item[Use Case Name] Student Karl
\item[Iteration] 0
\item[Summary] 
Karl ist Student an einer Berliner Uni.
Karl h�rt gerade eine Datenbankvorlesung und muss relationale Algebra lernen.
Karl will reval benutzen um  relationale Algebra zu lernen und eine
�bungen zu �berpr�fen.
\item[Preconditions]
Karl muss ghci und reval installiert und gestartet haben.
\item[Basic course of events]
Karl schreibt mit einem Texteditor die Tabellendeffinition vom
Uebungsblatt in ein Textdatei.
Nun kann er diese in ghci laden und anzeigen lassen.
Um nun seine Loesungen zu ueberpruefen kann er Ausdruecke in ghci
eingeben und rauswerten lassn.
Er kann die Ausdruecke in Haskell Syntax notieren oder eine alternative
mathematische Notation verwenden.
\item[Alternative paths]
none.
\item[Postconditions]
none.
\item[Date]
Daniel, Fabian- 23. Juni 2007
\end{description}

\subsection{Dozent}
\begin{description}
\item[Use Case Name] Dozentin Tina 
\item[Iteration] 0
\item[Summary] 
Tina ist Dozention fuer Datenbank-Technologien an einer Hamburger Uni.
Sie will das ihre Studenten relationale Algebra verstehen.
\item[Preconditions]
Tina muss ghci und reval installiert und gestartet haben.
\item[Basic course of events]
s. Student Karl
\item[Alternative paths]
none.
\item[Postconditions]
none.
\item[Date]
Daniel, Fabian- 23. Juni 2007
\end{description}

\section{spec}
Iteration: 0\\
Date: 23. Juni 2007
\subsection{Required}
\begin{itemize}
\item ADT f�r Tabelle 
\begin{itemize}
\item Notation f�r Tabellen bzw. Mengen
\item Notation f�r atomare Werte
\end{itemize}
\item Support f�r die sechs Basisoperationen
\begin{itemize}
\item Projektion
\item Selektion
\item Umbenennung
\item Kreuzprodukt
\item Vereinigung
\item Differenz
\end{itemize}
\item gute Code Qualit�t (stabil, les- und wartbar)
\item test-driven-development (nicht-funktional)
\end{itemize}
\subsection{Optinal}
\begin{itemize}
\item ADT f�r Tabelle 
\item Tabellen mit unterschiedlichen Typen f�r Spalten
\item gut lesbare ASCII-Art show Funktion f�r ADT
\end{itemize}
\subsection{Nice to have}
\begin{itemize}
\item relativ gute Performance (nicht funktional)
\item Ausgabe der Rechenschritte
\item einlesen der Tabellen aus txt-dateien
\item einlesen von befehlen aus txt-dateien
\item intuitive nicht-haskell Notation f�r die rel. Algebra
\end{itemize}
\end{document}
