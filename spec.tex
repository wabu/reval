% $Id: spec.tex,v 1.1 2007/06/23 15:33:52 fb Exp $
\documentclass[11pt,a4paper]{article}
\usepackage{a4}
\usepackage{ngerman}                            % neue Rechtschreibung
\usepackage[latin1]{inputenc}
%\usepackage[utf8-tes]{inputenc}
%\usepackage[applemac]{inputenc}
\usepackage{mathptmx}
\usepackage{amsmath}
\usepackage{latexsym}
%\usepackage{time}
\usepackage[ngerman]{babel}
\usepackage{lastpage}
\usepackage{alltt} %verbatim
% code in Latex anzeigen lassen
\usepackage{listings}
\usepackage{textcomp}
\usepackage{graphicx}
\usepackage{url}
\lstset{numbers=left}%, numberstyle=\tiny, numbersep=2pt}
%\lstset{language=Perl}
\lstset{language=Sql}
%
\DeclareMathSizes{11}{19}{13}{9}
%
\setcounter{secnumdepth}{5}                     % Ueberschriftentiefe
\title{DBS - Spec f�r DBS Project}
\author{Fabian Bieker (3981662), Daniel W�ber (4049590)}
\date{Version \$Id\$ - \today}
%
\begin{document}
%
\maketitle                      % Titel
%\tableofcontents                % Inhaltsverzeichnis
%\newcounter{zaehler}
%\stepcounter{section}
\begin{center}
  \textbf{Tutor: B�se}
\end{center}
%
\section{usecases}
\subsection{Student}
Karl ist Student an einer Berliner Uni.
Karl h�rt gerade eine Datenbankvorlesung und muss relationelle Algebra lernen.
\subsection{Profesor}

\section{spec}
\subsection{Required}

\subsection{Optinal}

\subsection{Nice to have}

\end{document}
